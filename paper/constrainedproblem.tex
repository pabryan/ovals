\label{section two}

We begin by stating the problem precisely.    Given a smooth, regular curve \(\gamma \in C^{\infty}(\So \to \RR^2)\) with length \(L(\gamma) = 2\pi\), we define the energy
\begin{equation}
\label{eq:Egamma}
E[\gamma] =% \inf\left\{\int_{\So} \abs{\partial_s f}^2 + \kappa^2 f^2 ds : \int_{\So} f^2 = 1\right\} 
 \inf\left\{\frac{\int_{\So} \abs{\partial_s f}^2 + \kappa^2 f^2 ds}{\int_{\So} f^2 ds} : f \not\equiv 0\right\}
\end{equation}
where \(f \in C^{\infty}(\So \to \RR^2)\) is assumed smooth.

For a given \(f \in C^{\infty}(\So \to \RR^2)\), let us also write
\begin{equation}
\label{eq:Egammaf}
R(\gamma, f) = \frac{\int_{\So} \abs{\partial_s f}^2 + \kappa^2 f^2 ds}{\int_{\So} f^2 ds}
\end{equation}
so that
\begin{equation}
\label{eq:EgammaEgammaf}
E(\gamma) = \inf\left\{R(\gamma, f) : f \not\equiv 0\right\}.
\end{equation}

Standard arguments imply that the energy is the first eigenvalue of the Euler-Lagrange equation,
\begin{equation}
\label{eq:ELf}
-\partial_s^2 f + \kappa^2 f = E[\gamma]f
\end{equation}
and that since $\kappa$ is continuous, the infimum is attained by the first eigenfunction \(f\) satisfying \(f > 0\). Therefore, we restrict our attention to \(f > 0\).

Let us write
\begin{equation}
\label{eq:TN}
T = \gamma', \quad N = J T
\end{equation}
\margincomment{I've messed with $J$ anti/clockwise, check all that. Changed it back}
for the unit tangent and normal, where $J$ is counter-clockwise rotation by $\pi/2$. % Here we choose the orientation \(N = -JT\) for convenience when dealing with circles later. 
Define the geodesic curvature \(\kappa\) by the Frenet-Serret formula
\begin{equation}
\label{eq:FS}
\partial_s T =  \kappa N, \quad \partial_s N = -\kappa T.
\end{equation}

For any \(\gamma\) and \(f > 0\), define
\begin{equation}
\label{eq:X}
X(s) %= X(f, \gamma)(s) 
= f(s) T(s).
\end{equation}
Note that we then have \( \> X(s) \ne 0   \) and
\begin{equation}
\label{eq:TNX}
T = \frac{X}{\abs{X}}, \quad %f = \abs{X}, 
N = J \frac{X}{\abs{X}}
\end{equation}
and we may recover \(\gamma\) and \(f\) 
from \(X\) (up to rigid motion of \(\gamma\)) by defining
\begin{equation}
\label{eq:gammaX}
\gamma(s)% = \gamma (X) (s) 
= \int_0^s \frac{X}{\abs{X}} ds, \quad f = \abs{X}.
\end{equation}
Notice that in this setting we cannot consider all $X:\So\to \R^2\setminus\lbrace0\rbrace$, since not all such $X$ will correspond to a closed curve $\gamma$.  We require an additional constraint on $X$: set
\begin{equation}
\label{eq:alpha}
\alpha(X)= \int_{\So} \frac{X}{\abs{X}} ds,
\end{equation}
then set
\[
\mathcal{C}:= \lbrace X \in C^{\infty}(\So\to \R^2\setminus\lbrace0\rbrace) :  \alpha(X)=0\rbrace.
\]
For $X\in\mathcal{C}$, $$0=\int \frac X{|X|}\,ds=\int T\,ds=\int \gamma'(s)\,ds= \gamma(2\pi)-\gamma(0),$$ and so $\gamma$ is a closed curve.

%Notice that defining the \(\RR^2\)-valued functional,
%\begin{equation}
%\label{eq:alpha}
%\alpha(X) = \int_{\So} \frac{X}{\abs{X}} ds
%\end{equation}
%we have that the set of \(X = X(\gamma, f)\) with \(f > 0\) satisfies,
%\[
%\alpha(X) = \int_{\So} T ds = 0.
%\]

Using the Frenet-Serret equations \eqref{eq:FS}, we may then express \(E(\gamma, f)\) in terms of \(X\) by
\begin{equation}
\label{eq:gammafX}
E(\gamma, f) = \frac{\int_{\So} |\partial_s f|^2 + \kappa^2 f^2 ds}{\int_{\So} f^2 ds} = \frac{\int_{\So} \abs{X'}^2 ds}{\int_{\So} \abs{X}^2 ds} = \frac{\|X'\|_2^2}{\|X\|_2^2} =: E(X).
\end{equation}

Then we have
\begin{equation}
\label{eq:infimums}
\begin{split}
\lambda = \inf\left\{E(\gamma) : L(\gamma) = 2\pi\right\} &= \inf\left\{R(\gamma, f) : L(\gamma) = 2\pi, f \not\equiv 0\right\} \\
&= \inf\left\{E(X) :  X\in\mathcal{C} %\forall s \> X(s) \ne 0, \alpha(X) = 0
\right\}.
\end{split}
\end{equation}
From this perspective, \Cref{main} is precisely the statement \(\lambda \geq 1\) with equality if and only if \(\gamma\) is an oval.


%For this reason, we call \(\alpha\) the \emph{constraint functional}. We have the following conjecture regarding the least energy for the %constrained problem:

%\begin{conj}
%\label{conj:main}
%\[
%\lambda = 1.
%\]
%\end{conj}


The \emph{unconstrained} problem for \eqref{eq:gammafX} is very simple:  it has Euler-Lagrange equation
$$X''+ E[X] X=0$$
which has solutions 
\begin{equation}  \label{ellipses}
\begin{cases}
X_0=V, %where $V_1\in\R^2$ is constant, 
\text{ with energy }E[X_0]=0; \\ 
%  $X_1=V_1\cos s+ V_2\sin s$  where $V_i\in\R^2$ is constant and $E[X_1]=1$; 
X_k=V_1\cos ks+ V_2\sin ks \text{ with energy }E[X_k]=k^2, \quad k=1,2,\dots
\end{cases}
\end{equation}
where $V_i\in\R^2$ is constant.     All these solutions except $X_0$ actually satisfy the constraint.  

The solutions of the form $X_1$ are ellipses, and are the conjectured minimisers with energy $E=1$.  They correspond to curves $\gamma$ given by ovals, including circles when $V_1$ and $V_2$ are orthonormal.    The solutions $X_k$ for $k>1$ are multiple coverings of ellipses.

Given the simplicity of the unconstrained Euler-Lagrange problem, all the dynamics of this problem are contained in the constraint set \(\C\). Crucial to our analysis, \Cref{constraint_submanifold} below states that \(\C\) is in fact a sub-manifold.
