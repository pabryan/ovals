Let $V:\So\to\R^2$ be a variation of $X$.  We calculate
\[
d\alpha_X (V) = \int_{\So} \frac{\pi_{\perp} (V)}{|X|} ds,
\]
where $\pi_\perp$ is projection on to the normal, $\pi_\perp{V}= \langle V,N\rangle N$.  

Defining \(\rho = \inpr{V}{N}\), we may rewrite this as
\[
d\alpha_X (V) = \int_{\So} \frac{\rho N}{|X|} ds.
\]
Working with respect to the angular parameter \(\theta(s)\) determined uniquely by
\[
N(s) = \cos \theta(s) e_1 + \sin \theta(s) e_2 \quad %\Rightarrow 
\text{and therefore }
T(s) = -\sin \theta(s) e_1 + \cos \theta(s) e_2,
\]
and satisfying
\[
\partial_s \theta = \kappa \quad \Rightarrow \quad \partial_s = \frac{1}{\kappa} \partial_{\theta}, \quad ds = \kappa d\theta,
\]
we have
\[
d\alpha_X (V) = \int_{\So} \bar{\rho} \cos\theta e_1 + \bar{\rho} \sin\theta e_2 d\theta
\]
where
\[
\bar{\rho} = \frac{\rho}{f \kappa}.
\]
%This follows from
%\[
%f = |X|, \quad d\theta = \kappa ds.
%\]

Then \(V\) infinitesimally preserves \(\alpha\) if and only if
\[
\bar{\rho} \perp_{L^2(d\theta)} \{\sin, \cos\}.
\]
%so that
%\[
%\bar{\rho} = c_0 + \sum_{n\geq 2} a_n \sin(n \theta) + b_n \sin(n \theta).
%\]

In other words, the tangent space, \(T_X \C\) to the constraint manifold \(\C\) at \(X\) is given by
\[
T_X \C = \{\mu T + f\kappa\bar{\rho} N : \mu \text{ arbitrary }, \bar{\rho} \perp_{L^2} \{\sin, \cos\}\}.
\]


%\begin{comment}%%%%%%%%%%%%%%%
%For \(\bar{\gamma} = \cos\theta s + \sin \theta s\) a circle, we have \(\bar{\kappa} \equiv 1\) and also eigenfunction, \(f \equiv 1\). Therefore \(\bar{X} = \bar{f} \bar{T}\) is the circle map,
%\[
%\bar{X}(s) = -\sin\theta e_1 + \cos\theta e_2.
%\]
%Moreover, \(\theta = s + \text{const}\) and we choose \(\text{const} = 0\) so that \(\theta = s\) and
%\[
%\bar{N}(\theta) = \cos\theta e_1 + \sin\theta e_2.
%\]
%Then we also have,
%\[
%\rho = \bar{\rho}
%\]
%and arbitrary constraint preserving variations may be written,
%\[
%V(\theta) = \bar{\mu}(\theta) T(\theta) + \bar{\rho}(\theta) N(\theta) = (- \bar{\mu}\sin\theta + \bar{\rho} \cos\theta) e_1 + (\bar{\mu}\cos\theta + \bar{\rho}\sin\theta) e_2
%\]
%where \(\bar{\mu} : \So \to \RR\) is arbitrary and \(\bar{\rho} \perp_{L^2} \{\sin, \cos\}\). Hence,
%\[
%T_{\bar{X}} \C = \{\bar{\mu} T + \bar{\rho} N : \bar{\mu} \text{ arbitrary }, \bar{\rho} \perp_{L^2} \{\sin, \cos\}\}.
%\]

%\end{comment}%%%%%%%%%
