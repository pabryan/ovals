% Created 2017-06-30 Fri 01:26
\documentclass[12pt]{article}
                           \usepackage[all]{pabmacros}
\usepackage[utf8]{inputenc}
\usepackage[T1]{fontenc}
\usepackage{fixltx2e}
\usepackage{graphicx}
\usepackage{longtable}
\usepackage{float}
\usepackage{wrapfig}
\usepackage[normalem]{ulem}
\usepackage{textcomp}
\usepackage{marvosym}
\usepackage[nointegrals]{wasysym}
\usepackage{latexsym}
\usepackage{amssymb}
\usepackage{amstext}
\usepackage{hyperref}
\tolerance=1000
\usepackage{amsmath}
\date{}
\title{Ovals Problem Proof}
\hypersetup{
  pdfkeywords={},
  pdfsubject={},
  pdfcreator={Emacs 24.5.1 (Org mode 8.2.10)}}

\DeclareMathOperator{\HS}{\mathcal{H}}
\DeclareMathOperator{\HSnc}{\HS_0}
\DeclareMathOperator{\C}{\mathcal{C}}
\DeclareMathOperator{\So}{\mathbb{S}^1}
\newcommand{\inpr}[2]{\ensuremath{\langle{#1},{#2}\rangle}}
\DeclareMathOperator{\G}{\mathcal{G}}
\DeclareMathOperator{\D}{D}
\DeclareMathOperator{\EL}{\mathcal{L}}

\begin{document}

\maketitle

\section{Class of closed curves}
\label{sec-1}

Let us write
\[
\HS = W^{1,2}(\So \to \RR^2)
\]
and
\[
\HSnc = \HS \setdiff \lbrace \text{constants} \rbrace
\]
where \(\lbrace \text{constants} \rbrace\) denotes the constant maps \(\So \to \RR^2\).

For \(X \in \HSnc\), and \(V \in \RR^2\), write
\[
X_V = X + V \in \HSnc
\]
and define
\[
\alpha_X : V \in \RR^2 \to \frac{1}{2\pi} \int \frac{X_v}{|X_V|} \in \RR^2.
\]

Define the class of curves,
\[
\C = \lbrace X \in \HSnc : \int_{\So} \frac{X}{|X|} ds = 0 \rbrace = \lbrace X \in \HSnc : \alpha_X((0, 0)) = 0 \rbrace.
\]

The factor \(1/2\pi\) in the definition of \(\alpha_X\) has no effect on the class \(\C\) but is a convenient normalisation in the following lemma.

\begin{lemma}
\label{lem:translate_C}
Let \(X \in \HSnc \backslash \lbrace 0 \rbrace\). Then there exists a \(V_0 \in \RR^2\) such that \(X_{V_0} \in \C\).
\end{lemma}

\begin{proof}
First observe that since \(X/|X|\) has unit length, \(\alpha_X\), maps to the closed unit disc, \(\bar{D} \subseteq \RR^2\).

\textbf{verify continuity}

Next, the map \(\alpha_X\) is continuous and for any \(r, \varphi\) we have
\[
\lim_{r \to \infty} \alpha_X (r e^{i\varphi}) = \frac{1}{2\pi} \int \frac{X + re^{i\varphi}}{|X + r^{i\varphi}|} ds = e^{i\varphi}.
\]

Moreover, for any curve \(\sigma(r)\) with \(\sigma\) asymptotic to \(r e^{i\theta}\) as \(r \to \infty\), we also have \(\lim_{r\to\infty} \sigma(r) = e^{i\theta}\). Thus the map \(\alpha_X\) extends continuously to the circle at infinity with \(\alpha_X(\infty e^{i\varphi}) = e^{i\varphi}\). Identifying \(\RR^2\) with the open unit disc, \(D\) we thus obtain a continuous map
\[
\alpha_X : \bar{D} \to \bar{D}
\]
which is the identity on the boundary.

But then \(\alpha_X(V_0) = (0, 0)\) for some \(V_0 \in D\) since otherwise the map \(V \mapsto \alpha_X(V)/|\alpha_X(V)|\) would define a retraction of \(D\) onto \(\So\) which is impossible.
\end{proof}

We will also need the following lemma:

\begin{lemma}
\label{lem:projection_C}
For all \(X \in \HSnc\), there exists a \(V \in \RR^2\) such that
\[
\int \langle X, V \rangle ds = 0.
\]
\end{lemma}

\begin{proof}
The map
\[
\beta_X : \varphi \in \RR \to \int \langle X, e^{i\varphi} \rangle ds
\]
is a continuous map such that
\[
\beta_X(\varphi + \pi) = -\beta_X(\varphi)
\]
and hence there is a \(\varphi \in \RR\) such that \(\beta_X(\varphi) = 0\). Then we take \(V = e^{i\varphi}\).
\end{proof}

\section{Min-max Construction}
\label{sec-2}

Let \(X \in \HSnc\) and \(V_0 \in \RR^2\) such that \(X_{V_0} \in \C\). The existence of \(V_0\) is guaranteed by Lemma \ref{lem:translate_C}. We define the following family of curves,
\[
\G_X = \lbrace v \in C^{\infty}(\RR \to \RR^2) : \exists t_0 \in \RR \quad v(t_0) = V_0, \quad \lim_{|t| \to \infty} |X_{v(t)}| = \infty \rbrace.
\]

For \(X \in \HSnc\) and \(v \in G_X\), let
\[
\lambda(X, v) = \sup \lbrace E(X_{v(t)}) : t \in \RR \rbrace = \sup \lbrace E(X + v(t)) : t \in \RR \rbrace.
\]

We have that \(\lambda(X, v)\) is finite since \(X_{v(t)}' = X'\) and hence
\[
\lim_{|t| \to \infty} E(X_{v(t)}) = \lim_{|t| \to \infty} \frac{\int |X'|^2 ds}{\int |X_v(t)|^2 ds} = 0
\]
as the definition of \(\G_X\) implies \(|X_v(t)| \to \infty\) as \(|t| \to \infty\).

Also from the definition of \(G_X\) we have for each \(v \in G_X\), the existence of a \(t_0\) such that \(X_{v(t_0)} = X_{V_0} \in \C\). Therefore
\[
\lambda(X, v) \geq E(X_{v(t_0)}) = E(X_{V_0}) \geq \lambda = \inf\lbrace E(X) : X \in \C\rbrace
\]
since \(X_{V_0} \in \C\) by definition of \(V_0\). Note that ovals have energy \(E = 1\) and Benguria-Loss ensures \(\lambda \geq 1/2\). Thus we have \(\lambda \in [1/2, 1]\).

Finally, define the min-max quantity,
\[
c = \inf \lbrace \lambda(X, v) : X \in \HSnc, v \in \G_X \rbrace.
\]
As noted, for any \(X \in \HSnc, v \in G_x\), we have \(\lambda(X, v) \geq \lambda\) and hence
\[
c \geq \lambda \geq \frac{1}{2}.
\]

It's also worth noting that \(E(X_{v(t)}) = 0\) for constant maps \(X\) and any \(v \in G_X\), which is why we work with \(\HSnc\) rather than with \(\HS\).

\begin{lemma}
\label{lem:E_bounded_by_c}
For all \(X \in \C\),
\[
E(X) \geq c.
\]
\end{lemma}

\begin{proof}
Let \(X \in \C\), fix \(W \ne (0, 0) \in \RR^2\) and define
\[
v(t) = t W.
\]
Then \(v \in \G_X\) since letting \(V_0 = v(0) = (0, 0)\) we have \(X + v(0) = X \in \C\) and obviously \(|v(t)| = |t| |V| \to \infty\) as \(|t| \to \infty\).

Thus we have a variation \(X_{v(t)}\) of \(X\) with \(X_{v(0)} = X\) and \(\partial_t X_{v(t)} = W\). The first variation of \(E\) is
\[
\partial_t E(X_{v(t)}) = \frac{2\|X_{v(t)}'\|_{L^2}^2}{\|X_{v(t)}\|_{L^2}^4} \int \inpr{X_{v(t)}}{W} ds.
\]
The second variation of \(E\) is
\[
\partial_t^2 E(X_{v(t)}) = \frac{8\|X_{v(t)}'\|_{L^2}^2}{\|X_{v(t)}\|_{L^2}^4} \left(\int \inpr{X_{v(t)}}{W} ds\right)^2 - \frac{2\|X_{v(t)}'\|_{L^2}^2}{\|X_{v(t)}\|_{L^2}^4} \int |W|^2 ds.
\]
At a critical point
\[
\frac{2\|X_{v(t)}'\|_{L^2}^2}{\|X_{v(t)}\|_{L^2}^4} \int \inpr{X_{v(t)}}{W} ds = 0
\]
and hence
\[
\partial_t^2 E(X_{v(t)}) = - \frac{2\|X_{v(t)}'\|_{L^2}^2}{\|X_{v(t)}\|_{L^2}^4} \int |W|^2 ds < 0.
\]
Thus any critical point is a local maximum and there can be at most one maximum. We also have that \(E(X_{v(t)}) \to 0\) as \(|t| \to \infty\) hence there exists a global maximum, which occurs at the unique critical point.

Now, by Lemma \ref{lem:projection_C}, there exists a \(V_0\) such that
\[
\int \inpr{X}{V_0} ds = 0.
\]
Choosing \(W = V_0\) we see that
\[
\partial_t|_{t=0} E(X_{v(t)}) = \frac{2\|X'\|_{L^2}^2}{\|X\|_{L^2}^4} \int \inpr{X}{V_0} ds = 0
\]
and hence \(t = 0\) is critical point and the maximum occurs at \(t = 0\). Hence
\[
c \leq \lambda(X, v(t) = X + tV_0) = \sup_{t \in \RR} E(X_{v(t)}) = E(X).
\]
\end{proof}

\section{The Palais-Smale Condition and a Mountain Pass Theorem}
\label{sec-3}

The energy \(E\) has some undesirable properties that prevent us from characterising minimsers by employing the direct calculus of variations, or indeed obtaining the Palais-Smale condition.

First observe that \(E\) is not coercive, since \(E\) is scale invariant, thus while \(\lim_{r\to \infty} \|rX\|_{W^{1,2}} = \infty\), \(E(r X) = E(X)\) remains bounded. Thus we encounter difficulties in employing the direct method of the calculus of variations.

The next problem faced is that the energy \(E\) is not Frechet differentiable since the map \(X \mapsto \|X'\|_{L^2}^2\) is not Frechet differentiable in \(W^{1,2} (\So, \RR^2)\).

\textbf{details}

Nevertheless, we are able to prove the main theorem.

\begin{theorem}
\[
\lambda = 1 = E(\text{ovals}).
\]
\end{theorem}

\begin{proof}
Let \((X_n, v_n)\) be a minimising sequence for the min-max quantity, \(c\):
\[
\lim_{n \to \infty} \lambda(X_n, v_n) = c.
\]
Also let \((t_n)\) be a sequence of real numbers with
\[
\lim_{n\to \infty} E(X_n + v_n(t_n)) = c.
\]

Define
\[
Y_n = \frac{X_n + v_n(t_n)}{\|X_n + v_n(t_n)\|_{L^2}}.
\]
Then \(\|Y_n\|_{L^2} = 1\) and
\[
\|Y_n'\|_{L^2}^2 = E(Y_n) = E(X_n + v(t_n)) \in [c-\tfrac{1}{n}, c + \tfrac{1}{n}]
\]
by passing to a subsequence if necessary.

Thus
\[
\|Y_n\|_{W^{1,2}} = \|Y_n\|_{L^2} + \|Y_n'\|_{L^2} \in [c+1-\tfrac{1}{n}, c + 1 + \tfrac{1}{n}]
\]
is uniformly bounded and hence there exists a weakly convergence subsequence
\[
Y_n \overset{W^{1,2}}{\rightharpoonup} Y
\]
as \(n \to \infty\). That is, for all \(V \in W^{1,2}\) we have
\[
\lim_{n\to \infty} \inpr{Y_n}{V}_{L^2} \to \inpr{Y}{V}_{L^2}, \quad \lim_{n\to \infty} \inpr{Y_n'}{V'}_{L^2} \to \inpr{Y'}{V'}_{L^2}.
\]

\textbf{Need to prove that \(\D_{Y_n} E \to 0\) in \(W^{-1,2}\) - Deformation Lemma!}

Using the fact that \(\D_{Y_n} \to 0\) in \(W^{-1,2}\), \(\|Y_n\|_{L^2} = 1\), and that \(\lim_{n\to\infty} E(Y_n) = c\), we have for every \(V \in W^{1,2}\) that
\[
\begin{split}
\inpr{Y'}{V'}_{L^2} - c \inpr{Y}{V}_{L^2} &= \lim_{n\to \infty} -\frac{2}{\|Y_n\|_{L^2}^2} \left[\inpr{Y_n'}{V'}_{L^2} - E(Y_n) \inpr{Y_n}{V}_{L^2}\right] \\
&= \lim_{n\to \infty} \D_{Y_n} (V) = 0
\end{split}
\]
as \(n \to \infty\). That is \(Y\) is a weak solution of
\[
\EL_c(Y) := Y'' + c Y = 0.
\]

But solutions of this equation are quantised, \(c = 0, 1, 2, \cdots\) with
\[
Y(s) = A_1 e^{i \sqrt{c} s} + A_2 e^{-i \sqrt{c} s}
\]
and we observed that \(c \geq 1/2\). Since the ovals have \(E = 1\), we may also assume that \(c \leq 1\), and hence quantisation implies \(c = 1\). Therefore
\[
Y(s) = A_1 e^{is} + A_2 e^{-is}
\]
has
\[
c = E(Y) = 1.
\]

Finally, by Lemma \ref{lem:E_bounded_by_c} we have
\[
\forall X \in \C, E(X) \geq c = 1
\]
and hence \(\lambda = 1\).
\end{proof}
% Emacs 24.5.1 (Org mode 8.2.10)
\end{document}