\documentclass[12pt, a4paper]{amsart}



\address{ }
\usepackage[left=3cm,top=2.5cm,right=3cm]{geometry}
\usepackage{parskip, color}
\usepackage[normalem]{ulem}
%\date{letterdate} if you want to fix the date on the letter; otherwise the date will default to the current date when the letter is printed.

\usepackage{geometry}

%\usepackage{lmodern}
\usepackage[T1]{fontenc}


\usepackage{versions}
%%%%%%%%%
\includeversion{lecturenotes}  \excludeversion{outline}
%\excludeversion{lecturenotes}   \includeversion{outline}
%%%%%%%%%

%\usepackage{kpfonts}

%\usepackage[T1]{fontenc}
%\usepackage{cmbright}
 
 \usepackage{hyperref}
 
\newcommand{\margincomment}[1]{\marginpar{\footnotesize{#1}}}
\newcommand{\pd}[2]{\dfrac{\partial#1}{\partial#2}}
\DeclareMathOperator{\divergenz}{div}
\newcommand{\bigR}{{\mathbb R}}

\newcommand{\x}{\lVert x \rVert}

\newcommand{\eps}{\epsilon}

\newcommand{\emptynorm}{\lVert \cdot \rVert}

\usepackage{setspace}

\usepackage{enumitem}

\usepackage{pdfpages}

\theoremstyle{remark}
\newtheorem*{definition}{Definition}
\newtheorem*{exercise}{Exercise}



\begin{document}

  
\spacing{1}




\title{  }
%\maketitle

Dear Mat,

I'm pretty sure this is not what you are asking, but at any rate:

\textbf{The ovals, in terms of $f$, $\kappa$}.
In this case we write for two $v_1,v_2\in\bigR^2$
$$X(s)=\cos(s) v_1+\sin(s) v_2. $$
(Actually we'd be better off writing $X=a\cos(s) e_1 + b\sin(s) e_2$...)

Then
$$X_s(s)=-\sin(s)v_1+\cos(s)v_2$$
and
$$X_{ss}(s)=-X.$$

Relation to graph quantities:  Recover the curve itself by
$$\gamma(s)=\int_0^s \frac{X(\sigma)}{|X(\sigma)|}\,d\sigma$$
but also
$$\text{tangent }   T(s)=\frac{X(s)}{|X(s)|}$$
$$\text{ normal  }   N(s)=JT =\frac{JX(s)}{|X(s)|}$$
$$ \text{eigenfunction } f=|X| = \left( \cos^2(s) |v_1|^2 + 2\sin(s)\cos(s)\langle v_1,v_2\rangle + \sin^2(s)|v_2|^2\right)^{1/2}  $$
To find curvature, remember that $X_s=f'T+ \kappa f N$ so $\kappa f=\langle X_s,N\rangle$ and then
\begin{align*}
\text{curvature } \kappa&=\frac{\langle X_s,N\rangle}{|X|}\\
&=\frac{\langle X_s,JT\rangle}{|X|}\\
&=\frac{\langle X_s,JX\rangle}{|X|^2}\\
\end{align*}
So putting in our explicit form of $X$ here we find
\begin{align*}
{\langle X_s,JX\rangle}&=\langle -\sin(s)v_1+\cos(s)v_2, \cos(s) J(v_1) +\sin(s) J(v_2)\rangle \\
&=-\sin^2(s)\langle  v_1,J v_2\rangle + \cos^2(s)\langle v_2,Jv_1\rangle \\
\intertext{but antisymmetry $\langle Ja,b\rangle = -\langle a,Jb\rangle$ }
&= \langle v_2,Jv_1\rangle =C
\end{align*}
Thus 
$$\kappa=\frac{C}{|X|^2}= \frac{C}{f^2}.$$
(Is it true that $\langle X_s, JX\rangle=C|X|^2$ characterises ovals?)


Julie
\bigskip

Julie: Yes, that's what I'm after. I wanted to check that (and if so, in what sense)
\[
\Big(|X_s|^2-|X|^2=\Big)\quad f_s^2+\kappa^2 f^2-f^2=0
\]
on the ovals$^{\mathrm{tm}}$. This doesn't seem to be true leading to a contradiction which I can't seem to resolve.

Recall that (\emph{Much ado about ovals}, Lemma 5.3)
\[
\psi\cdot (|X_s|^2-|X|^2)
\]
is constant (in $s$) for any $X$ and any $\psi$ tangent to M\"ob$(S^1)$ at the identity\footnote{I have checked this and checked it again. It seems to be watertight.}. If 
\[
E(X):=\int(|X_s|^2-|X|^2)ds=0
\]
then the integrand must be zero at at least one point and hence this constant is zero. Thus, it appears that the density has to be identically zero for any $X$ with $E(X)=0$. This is not \emph{quite} the case: Recall that $\psi_{sss}+4\psi_s=0$. I don't think that every $\psi$ satisfying this equation can be tangent to a family of M\"obius transformations. But at least the $\psi$'s should all have a discreet set of zeroes. Let $Z=\{s_1,\dots,s_k\}\subset [0,2\pi)$ be the set of mutual zeros of all the $\psi$'s (I suspect that $Z=\{0,\pi/2,\pi,3\pi/2\}$). Then
\begin{equation}\label{eq:density}
|X_s|^2-|X|^2\equiv 0\quad \text{in}\quad [0,2\pi)\setminus Z
\end{equation}
for any pair $X$ satisfying $E(X)=0$. The challenge is then to show that the only such $(\gamma,f)$ are the family of ovals$^{\mathrm{tm}}$. I was worried that something is wrong here, since a na\"ive argument gives
\[
|X_s|^2-|X|^2=f_s^2+\kappa^2 f^2-f^2\equiv 0\Rightarrow \kappa\leq 1\Rightarrow \kappa\equiv 1\Rightarrow f_s\equiv 0\,.
\]
Thus, the round circle with eigenfunction $f\equiv \mathrm{Const.}$ is the only solution with $\{f=0\}$ discreet and $f_s^2+\kappa^2 f^2-f^2\equiv 0$. So I wanted to check that \eqref{eq:density} holds on the ovals$^{\mathrm{tm}}$. Setting $\kappa=C/f^2$, we \emph{can} solve \eqref{eq:density} (assuming $\{0,\pi\}\subset Z$) with (for any $a\in\mathbb{R}$)
\[
f^2(s):=
\begin{cases}
C\cosh(a+2s)&\text{if}\quad s\in[0,\pi]\\
C\cosh(a+4\pi-2s)&\text{if}\quad s\in(3\pi/2,2\pi)\,.
\end{cases}
\]
Note that this $f$ is positive and continuous on $\mathbb{R}/2\pi\mathbb{Z}$. Thus, $\kappa$ is continuous. But it is not differentiable :(

The curve $\gamma$ can be reconstructed as
\[
\gamma(\theta)=\frac{2\pi}{L}\left(\int_0^\theta\frac{\cos\sigma\,d\sigma}{\kappa(\sigma)},\int_{\pi/2}^\theta\frac{\sin\sigma\,d\sigma}{\kappa(\sigma)}\right)\,,
\]
where
\begin{align*}
L:=\int_0^{2\pi}\frac{d\sigma}{\kappa(\sigma)}={}& \int_0^{\pi}\cosh(a+2\sigma)d\sigma+\int_\pi^{2\pi}\cosh(a+4\pi-2\sigma)d\sigma\\
={}&2\sinh \pi \cosh(a+\pi)
\end{align*}
%In that case, $\kappa(s)\to 1$ as $a\to\infty$ and $\kappa(s)\to 0$ as $a\to-\infty$ (except when $s\in\{0,\pi\}$ in which case $\kappa\to\infty$).

\textbf{Questions}
\begin{itemize}
\item[--] Are the ovals$^{\mathrm{tm}}$ smoother than $C^2$? 
\item[--] Is this the right $f=|X|^2$?
\item[--] What the hell is going on here?
\end{itemize}

\bigskip

Mat
\end{document}
