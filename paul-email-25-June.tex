  \documentclass[12pt, a4paper]{amsart}



\address{ }
\usepackage[left=3cm,top=3.5cm,right=3cm]{geometry}
\usepackage{parskip, color}
\usepackage[normalem]{ulem}
%\date{letterdate} if you want to fix the date on the letter; otherwise the date will default to the current date when the letter is printed.

\usepackage{geometry}

%\usepackage{lmodern}
\usepackage[T1]{fontenc}


\usepackage{versions}
%%%%%%%%%
\includeversion{lecturenotes}  \excludeversion{outline}
%\excludeversion{lecturenotes}   \includeversion{outline}
%%%%%%%%%

\usepackage{kpfonts}

%\usepackage[T1]{fontenc}
%\usepackage{cmbright}
 
 \usepackage{hyperref}
 
\newcommand{\margincomment}[1]{\marginpar{\footnotesize{#1}}}
\newcommand{\pd}[2]{\dfrac{\partial#1}{\partial#2}}
\DeclareMathOperator{\divergenz}{div}
\newcommand{\bigR}{{\mathbb R}}
\newcommand{\RR}{{\mathbb R}}

\newcommand{\x}{\lVert x \rVert}

\newcommand{\eps}{\epsilon}

\newcommand{\emptynorm}{\lVert \cdot \rVert}

\usepackage{setspace}

\usepackage{enumitem}

\usepackage{pdfpages}

\theoremstyle{remark}
\newtheorem*{definition}{Definition}
\newtheorem*{exercise}{Exercise}



\begin{document}

  
\spacing{1.1}




\title{ Paul email 25 June }
\maketitle

It seems, that a lot of what we need is in chapter 8 of Evans :)

Let's write $X: \mathbb{S}^1 \to \RR^2$ and $\Phi = \lbrace X_{\nu}
\rbrace$ for a continuous family. Then we define
\[
\mathcal{F} (\Phi) = \sup_{\nu} E(X_{\nu})
\]
and
\[
W = \inf_{\Psi \in [\Phi]} \mathcal{F} (\Psi)
\]
where $[\Phi]$ denotes the homotopy class of $\Phi$.

We take a min-max sequence $\Phi^n = \lbrace X_{\nu}^n \rbrace$ with
\[
\lim_{n\to\infty} \mathcal{F} (\lbrace X_{\nu}^n \rbrace) = W.
\]

Since our energy is invariant under scaling, we can assume that
$\|X_{\nu}^n\|_{L^2} = 1$ for all $\nu, n$ and so
\[
E(X_{\nu}^n) = \|\nabla X_{\nu}^n\|_{L_2}^2.
\]

So now by definition of $\mathcal{F}$ as the $\sup$ and by definition
of $W$ as the inf, we get a sequence $Y_n = (X^n_{\nu_n})$ in the unit
sphere of $L_2$ with
\[
E(Y_n) = \|\nabla Y_n\|_2^2 \to W.
\]

So we can assume that $\|Y_n\|_{W^{1,2}}$ is uniformly bounded
independently of $n$ and thus get a weakly convergent subsequence
\[
Y_n \rightharpoonup_{W^{1,2}} Y
\]

We have lower semi-continuity
\[
E(Y) = \|\nabla Y\|_2^2 \leq \liminf_{n\to\infty} \|\nabla Y_n\|_2^2 =
\liminf_{n\to\infty} E(Y_n) = W.
\]

Now if we start with $X$ and produce a continuous family and homotopy
class, I guess we also need to know that $E(X) \geq W$ to conclude
that $E(X) \geq W = E(Y)$. This will be true if our continuous family
has $E(X) \geq \mathcal{F}(\{X_{\nu})\). In other words, our
continuous family really does need to deform $X$ to decrease $E(X)$.
We know that translating by $V$ does decrease $E(X)$ as $V \to \infty$
but we need it to decrease $E(X)$ for all $V$.

Something to check.

Then, to finish we need to show that $Y$ is a critical point and that
$0 < W < 3/2$ and hence $Y$ is an oval. Showing $Y$ is a critical
point seems to be essentially the content of the min-max idea - that
it produces a Palais-Smale sequence producing a critical point. And
the homotopies (canonical family) should keep $\mathcal{F}$ bounded
below so that $Y$ is not the absolute minimiser (i.e. the point). The
upper bound is no problem - it's the lower bound for $W$ that we have
to worry about.

Thus If we construct our min-max sweepouts and homotopies correctly,
we should get that $Y$ is a critical point.

And perhaps Julie's topological ideas for $\alpha$ will allow us to
also conclude that in any of our homotopies of the canonical family,
we get something in $\mathcal{C}$. In fact, all we need is for
anything in the homotopy class, there is another element of the
homotopy class that includes something from $\mathcal{C}$ and with the
same $\mathcal{F}$. I think this is in fact the way it's done in the
papers. This would allow us to conclude the lower bound on $W$.

I haven't looked yet, but the math.stackexchange question linked to in
my last email suggests somewhere that the connection min-max <->
Palais-Smale is discussed. It seems to be the underlying idea in the
papers we've looked at, but they don't seem to clearly show how
min-max produces Palais-Smale sequences. Maybe we just need to read
them more closely.

Cheers,
Paul.


 \end{document}
