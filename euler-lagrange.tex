\documentclass[12pt]{article}

\usepackage[utf8]{inputenc}
\usepackage[T1]{fontenc}
\usepackage{fixltx2e}
\usepackage{graphicx}
\usepackage{longtable}
\usepackage{float}
\usepackage{wrapfig}
\usepackage[normalem]{ulem}
\usepackage{textcomp}
\usepackage{marvosym}
\usepackage[nointegrals]{wasysym}
\usepackage{latexsym}
\usepackage{amssymb}
\usepackage{amstext}
\usepackage{hyperref}
\tolerance=1000
\usepackage{amsmath}
\usepackage{amsthm}
\usepackage{color}

\DeclareMathOperator{\RR}{\mathbb{R}}
\DeclareMathOperator{\HS}{\mathcal{H}}
\DeclareMathOperator{\HSconst}{\mathcal{V}}
\DeclareMathOperator{\HSnoconst}{\mathcal{K}}
\DeclareMathOperator{\C}{\mathcal{C}}
\DeclareMathOperator{\So}{\mathbb{S}^1}
\newcommand{\inpr}[2]{\ensuremath{\left\langle{#1},{#2}\right\rangle}}
\DeclareMathOperator{\G}{\mathcal{G}}
\DeclareMathOperator{\D}{D}
\DeclareMathOperator{\EL}{\mathcal{L}}
\DeclareMathOperator{\setdiff}{\backslash}
\DeclareMathOperator{\trans}{\tau}
\DeclareMathOperator{\Id}{Id}

\newtheorem{thm}{Theorem}[section]
\newtheorem{lem}[thm]{Lemma}
\newtheorem{prop}[thm]{Proposition}
\newtheorem{cor}[thm]{Corollary}
\newtheorem{rem}[thm]{Remark}
\newtheorem{conj}{Conjecture}
\renewcommand{\theconj}{\Alph{conj}}


\title{Euler Lagrange Equation}
\author{}
\date{}

\begin{document}

\maketitle

\section{The Constrained Problem}
\label{sec:constrained}

Our space is \(\HS = W^{1,2} (\So \to \RR^2)\). We seek to minimise the Dirichlet energy,
\[
E(X) = \|X'\|_2^2 = \int_{\So} \abs{X'(s)}^2 ds
\]
subject to the constraints
\[
I(X) := \|X\|_2^2 = 1, \quad \alpha(X) := \int_{\So} \frac{X(s)}{\abs{X}(s)} ds = (0, 0).
\]

For \(\lambda \in \RR\), \(\Lambda \in \RR^2\) (Lagrange multipliers) define the functional
\[
F(X) = E(X) - \lambda I(X) + \inpr{\Lambda}{\alpha(X)}.
\]
Then constrained critical points of \(E\) are unconstrained critical points of \(F\). The minus sign is a convenience ensuring that for a constrained critical point \(X\) of \(F\), we have \(\lambda = E(X)\). A constrained minimiser, \(X\) of \(E\) is a critical point of \(F\) with \(I(X) = 1, \alpha(X) = (0, 0)\) and satisfying
\[
d^2 F_X (V, V) \geq 0
\]
for all \(V\) tangent to the constraint manifold \(\C = \{I(X) = 1, \alpha(X) = 0\}\) ({\color{red}Proof it's actually a manifold?}). Note that while \(X\) is a critical point of \(F\) so that \(dF_X (V) = 0\) for any \(V \in T_X\HS = \HS\), it is only assured to be non-negative definite for \(V \in T_X \C\). In fact, we know that \(E\) can be decreased by moving in ({\color{red} some, all?}) directions transverse to \(\C\).

\section{First Variation}
\label{sec:firstvar}

Denoting the \(\RR^2\) inner-product by \(\inpr{\cdot}{\cdot}\) and the \(L_2\) inner product for maps \(\So \to \RR^2\) by \(\inpr{\cdot}{\cdot}_2\), We have
\begin{align*}
dE_X (V) &= \inpr{-X''}{V}_2 \\
dI_X (V) &= \inpr{X}{V}_2 \\
d\alpha_X (V) &= \int_{\So} \frac{1}{\abs{X}(s)} \inpr{N(s)}{V} N(s) ds \\
dF_X (V) &= -\inpr{X'' + \lambda X}{V}_2 + \inpr{\Lambda}{\int_{\So} \frac{1}{\abs{X}(s)} \inpr{N(s)}{V} N(s) ds} \\
&= -\inpr{X'' + \lambda X - \frac{1}{\abs{X}} \inpr{\Lambda}{N}N}{V}_2
\end{align*}
where \(N(s) = J(T)\) with \(T = X/\abs{X}\) and \(J\) counter-clockwise rotation by \(\pi/2\) (we could of course choose the other normal by rotating clockwise so long as we remain consistent).

Thus we obtain the Euler-Lagrange equation satisfied by critical points of \(F\) (constrained or otherwise),
\begin{equation}
\label{eq:el}
X'' + \lambda X = \frac{1}{\abs{X}} \inpr{\Lambda}{N}N
\end{equation}
or equivalently, we have the (pointwise) orthonormal decomposition
\begin{equation}
\label{eq:pointwise_orth}
X'' = -\lambda \abs{X} T + \frac{1}{\abs{X}} \inpr{\Lambda}{N}N
\end{equation}

Substituting into the energy gives,
\begin{equation}
\label{eq:Ecrit}
\begin{split}
E(X) &= \int_{\So} \abs{X'}^2 ds = -\int_{\So} \inpr{X''}{X} ds \\
&= \int_{\So} \inpr{\lambda X - \frac{1}{\abs{X}} \inpr{\Lambda}{N}N}{X} ds = \lambda \inpr{X}{X}_2 = \lambda
\end{split}
\end{equation}
since \(\inpr{X}{N} = 0\) and \(\inpr{X}{X}_2 = I(X) = 1\). Thus \(\lambda = E(X)\) as promised.

Substituting into \(I\) gives
\begin{equation}
\label{eq:Icrit}
\begin{split}
1 = I(X) &= \int_{\So} \abs{X}^2 ds = \frac{1}{\lambda^2} \int_{\So} \abs{-X'' +  \frac{1}{\abs{X}} \inpr{\Lambda}{N}N}^2 ds \\
&= \frac{1}{\lambda^2} \int_{\So} \abs{X''}^2 - 2\frac{1}{\abs{X}} \inpr{\Lambda}{N}\inpr{X''}{N} + \frac{1}{\abs{X}^2} \inpr{\Lambda}{N}^2 ds.
\end{split}
\end{equation}

Substituting into \(\alpha\) gives,
\[
(0, 0) = \alpha(X) = \int_{\So} \frac{X}{\abs{X}} ds = \frac{1}{\lambda} \int_{\So} \frac{1}{\abs{X}} \left(-X'' + \frac{1}{\abs{X}} \inpr{\Lambda}{N} N\right) ds
\]
giving the identity for \(\Lambda\)
\begin{equation}
\label{eq:vecidentity_lambda}
\int_{\So} \frac{1}{\abs{X}^2} \inpr{\Lambda}{N} N ds = \int_{\So} \frac{X''}{\abs{X}} ds
\end{equation}

Taking the inner product with \(\Lambda\) gives
\[
\int_{\So} \frac{1}{\abs{X}^2} \inpr{\Lambda}{N}^2 ds = \int_{\So} \frac{1}{\abs{X}} \inpr{X''}{\Lambda} ds,
\]
which combined with equation \eqref{eq:Icrit} results in
\[
1 = \frac{1}{\lambda^2} \int_{\So} \abs{X''}^2 - 2\frac{1}{\abs{X}} \inpr{\Lambda}{N}\inpr{X''}{N} +  \frac{1}{\abs{X}} \inpr{X''}{\Lambda} ds.
\]
We rewrite this as an identity for \(\Lambda\):
\begin{equation}
\label{eq:Lambdaeqn}
\begin{split}
\inpr{\Lambda}{\frac{1}{\abs{X}} (X'' - 2 \inpr{X''}{N}N)}_2 &= \inpr{\int_{\So} \frac{1}{\abs{X}} (X'' - 2 \inpr{X''}{N}N) ds}{\Lambda} \\
&= \int_{\So} \frac{1}{\abs{X}} \inpr{X'' - 2 \inpr{X''}{N}N}{\Lambda} ds \\
&= E(X)^2 - \int_{\So} \abs{X''}^2 ds
\end{split}
\end{equation}
recalling that \(\lambda = E(X)\).


\end{document}